% !TEX program=xelatex

\documentclass[aspectratio=169]{beamer}

% У зависности од тога да ли желимо да пишемо на ћирилици или латиници, потребно је учитати тачно један од наредних модула пакета babel:
% 1) за ћирилицу: serbianc
% 1) за латиницу: serbian
\usepackage[serbianc]{babel}
% \usepackage[serbian]{babel}

% Стилови презентације
\usepackage{stilovi/beamerthemematf}
\usepackage{stilovi/beamerfontthemematf}

% Приказивање секције из садржаја пре сваког наслова
\AtBeginSection[]
{
    \begin{frame}
        \frametitle{Садржај}
        \tableofcontents[currentsection]
    \end{frame}
}

% Информације за насловну страницу 
\title{Наслов}
\author{предавач}
\date{\today}

% Почетак документа
\begin{document}

% Креирање насловне странице
\frame{\titlepage}

% Садржај
\begin{frame}
    \frametitle{Садржај}
    \tableofcontents
\end{frame}

\section{Секција 1}

% Први фрејм од 4 слајда
\begin{frame}
    \frametitle{поднаслов}

    Текст ван набрајања.

    \begin{itemize}
        \item<1-> Текст видљив од слајда 1
        \item<2-> Текст видљив од слајда 2
        \item<3> Текст видљив само на слајду 3
        \item<4-> Текст видљив од слајда 4
    \end{itemize}
\end{frame}

\section{Секција 2}

% Други фрејм од 3 слајда
\begin{frame}
    \frametitle{поднаслов}

    У овом слајду \pause

    текст ће бити парцијално видљив. \pause

    Коначно, све ће бити видљиво у последњем слајду.
\end{frame}

% Трећи фрејм од 1 слајда
\begin{frame}
    \frametitle{поднаслов}

    \begin{columns}
        \begin{column}{0.5\textwidth}
            Овај слајд има 2 колоне истих ширина \\
            који се састоји од:

            \begin{itemize}
                \item Текста са леве стране
                
                \item Слике са десне стране
            \end{itemize}
        \end{column}
        \begin{column}{0.5\textwidth}
            \begin{center}
                \includegraphics[width=0.5\textwidth]{resursi/slika.png}
            \end{center}
        \end{column}
    \end{columns}
\end{frame}

% Четврти фрејм од 1 слајда
\begin{frame}[fragile]
    \frametitle{поднаслов}

    Ово је пример једног изворног кода:

\begin{lstlisting}[language=C++]
Class x;
std::vector<Class> xs;
xs.push_back(x);
\end{lstlisting}
\end{frame}

\end{document}